\chapter*{Conclusion générale}
\addcontentsline{toc}{chapter}{Conclusion générale}
\markboth{Conclusion générale}{}


En guise de conclusion, notre projet de fin d’études réalisé au sein de Vermeg porte sur la conception et le développement d'une marketplace d’API nommée "InfinityAPI". Ce projet a pour objectif d'offrir une plateforme où les développeurs peuvent aussi bien proposer et monétiser leurs propres API que souscrire et consommer une large gamme d'API proposées.  \\
Pour la réalisation technique, nous avons adopté l’architecture MEAN pour le développement de la marketplace. Ce projet a été une véritable expérience enrichissante, nous permettant d’acquérir de nombreuses compétences.\\ Nous avons ainsi mieux maîtrisé les framewoks utilisés et découvert des biblithèques et des outils de développement, tels que Swagger pour la documentation des API et chartJS pour la réalisation de graphiques et de statistiques. Il était également intéressant d’apprendre à intégrer des méthodes de paiement telles que Stripe  et PayPal ou encore à gérer les notifications en temps réel.\\
De plus, cette expérience de stage nous a permis de découvrir la vie professionnelle et de développer nos compétences en organisation, en gestion du temps et en travail d'équipe.\\
À court terme, plusieurs améliorations peuvent être apportées à notre projet, telles que l'ajout d'un payout automatiquement, d'un forum de discussion, ainsi que la mise en place d'une politique de site et d'une gestion des rôles. \\ 
À plus long terme, nous pourrions ajouter la possibilité de réaliser des abonnements plutôt que des souscriptions sur les API. De plus, nous pourrions intégrer différents types d’API comme SOAP, RPC et GraphQL, ainsi que l’intelligence artificielle pour recommander des API en fonction des besoins spécifiques des utilisateurs et améliorer la recherche et la découverte d’API sur la plateforme. 