\chapter*{Introduction générale}
\addcontentsline{toc}{chapter}{Introduction générale} % to include the introduction to the table of content
\markboth{Introduction générale}{} %To redefine the section page head


%L'histoire de l'API remonte bien avant l'ère des solutions numériques mobiles contemporaines, plongeant ses racines dans les débuts de l'informatique. Originellement conçus pour relier des ordinateurs traditionnels, ces ensembles de programmation étaient jadis intégrés localement aux systèmes d'exploitation tels que Windows, Mac OS et Linux. Leur fonction initiale était de faciliter la communication entre les ordinateurs centraux ou macroordinateurs. \\Avec l'essor d'Internet, l'importance des solutions API s'est accrue exponentiellement, s'adaptant constamment à l'évolution du paysage numérique. En milieu professionnel, les API sont devenues des outils essentiels pour accélérer les processus de traitement de l'information, permettant l'automation des tâches comme l'extraction et le tri de contenus provenant de multiples sources. Cette interconnexion de données a permis à divers secteurs de gagner en efficacité, faisant des APIs un pilier fondamental des solutions digitales et un avantage concurrentiel majeur dans l'écosystème numérique contemporain. En simplifiant le travail de fond des développeurs, les APIs leur offrent un gain de temps considérable en réduisant la quantité de code qu'ils doivent créer, contribuant ainsi à une innovation plus rapide et à des cycles de développement plus courts.
De nos jours, dans un monde en constante évolution, les avancées technologiques bouleversent notre quotidien et transforment les pratiques des entreprises. La révolution numérique a introduit de nouvelles manières de communiquer, de travailler et de faire des affaires. Les interfaces de programmation d’applications (API) sont devenues des éléments essentiels de ce paysage technologique, facilitant l’intégration et l’interopérabilité entre divers systèmes et applications. \\
Parallèlement, les marketplaces d'API ont émergé comme des plateformes incontournables pour centraliser, simplifier l'accès et la monétisation des API.\\
C’est dans ce contexte que s’intègre notre projet de fin d’études qui a pour objectif de de concevoir et de développer une marketplace d’API. Cette plateforme facilitera la connexion entre les fournisseurs et les consommateurs d'API, offrant ainsi une interface centralisée pour découvrir, gérer et monétiser les APIs.  \\
Ce rapport détaille les différentes étapes nécessaires pour aboutir à la réalisation de cette marketplace d’API. Il se divise en plusieurs chapitres.\\ 
Le premier chapitre introduit l'entreprise, aborde les concepts clés relatifs aux API et enchaîne avec une étude de marché suivie d'une étude de l'existant. Il se conclut par une présentation de la méthodologie adoptée et du langage de modélisation utilisé. \\
Le deuxième chapitre est consacré à la planification du projet, la spécification des besoins, le backlog du produit, et l'environnement de développement. \\ 
Les trois derniers chapitres présentent les différents sprints du projet à travers le backlog du sprint, la spécifications fonctionnelle la conception et la réalisation .